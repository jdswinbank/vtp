\documentclass[a4paper,11pt]{ivoa}
\usepackage[a4paper,left=1.2in,right=1.2in,top=1.2in,bottom=1.2in]{geometry}
\usepackage{lmodern}
\usepackage{natbib}
\usepackage{minted}

\title{VOEvent Transport Protocol}
\author{Alasdair Allan, Robert B. Denny, John Swinbank}
\editor{John Swinbank}
\ivoatype{IVOA Proposed Recommendadation}
\ivoagroup{Time Domain Interest Group}
\version{1.2 (draft)}


\begin{document}
\maketitle

\section*{Abstract}

The IVOA VOEvent Recommendation \citep{Seaman:2011} defines a means of
describing a transient celestial event but, purposely, remains silent on the
topic of how those descriptions should be transmitted. This document
formalizes a TCP-based protocol for VOEvent transportation that has been in
use by members of the VOEvent community for several years and discusses the
topology of the event distribution network. It is intended to act as a
reference for the production of compliant protocol implementations.

\section*{Status of this document}

This document has been produced by John Swinbank based on the version 1.1
specification by Allan \& Denny. This draft version is a work in progress. It
is intended to add clarifications to the previous version without changing the
on-the-wire protocol.

This is an IVOA Note expressing suggestions from and opinions of the authors.
It is intended to share best practices, possible approaches, or other
perspectives on interoperability with the Virtual Observatory. It should not
be referenced or otherwise interpreted as a standard specification.

A list of current IVOA Recommendations and other technical documents can be
found at \url{http://www.ivoa.net/documents/}.

\section*{Acknowledgements}

\begin{itemize}
    \item{Robert W. White, ex LANL, now at NREL}
    \item{Phillip Warner, ex NOAO}
    \item{Robert Seaman, NOAO}
\end{itemize}

Swinbank acknowledges support from the European Research Council via Advanced
Investigator Grant 247295.

\newpage

\tableofcontents

\newpage

\section{Introduction}
\label{sec:intro}

The VOEvent standard \citep{Seaman:2011} defines a means of representing a
transient celestial event with an implicit request for action on the part of
the recipient. The VOEvent standard is transport neutral: it does not take a
position on the mechanism by which the event should be transmitted from its
author to interested recipients. However, it encourages the construction of “a
robust general-purpose network of interoperating brokers” for event
transmission.

To date, a number of different event distribution networks have been
prototyped and met with varying degrees of technical success and community
adoption. However, as the number of interested participants grows, and
next-generation large-scale survey instruments such as LSST\footnote{Large
Synoptic Survey Telescope; \url{http://www.lsst.org/}}, LIGO\footnote{Laser
Interferometric Gravitational-Wave Observatory; \url{http://www.ligo.org}},
LOFAR\footnote{Low Frequency Array; \url{http://www.lofar.org}} and
SKA\footnote{Square Kilometer Array; \url{http://www.ska-telescope.org/}} are
developed and begin to become available, it is clear that a standard,
interoperable mechanism for event communication is required. It is such a
mechanism that this document aims to describe.

The purpose of the protocol described herein is to transport a VOEvent message
from its sender to one or more interested recipients. To achieve this, we
envision three distinct network roles: authors, which create events; brokers,
which receive events from authors and distribute them, and subscribers, which
receive and (if appropriate) act upon the events. See Figure \ref{fig:network}
for an illustration. Note that a single entity may perform more than one role
within the network: for example, creating events and distributing its own
creations (combining the author and broker roles) or receiving events from a
broker and redistributing them to a list of subscribers (combining the
subscriber and broker roles).

Building upon this architecture, a strongly-connected set of brokers which
subscriber to each other's event streams and redistribute to their subscribers
(the ``VOEventNet backbone'') provides a fault-tolerant system which is
resilient against the failure of one or more network entities. Such a backbone
system is already under construction by members of the VOEvent community.  The
protocol described herein is intentionally as simple as possible while still
accomplishing the required task. There are some value-added VOEvent services
being developed which may require more complex protocols, but these fall
outside the scope of the current document.

\section{Terminology}

Throughout this document, we adopt the terminology of RFC 2119
\citep{Bradner:1997}. In particular, note that:

\begin{itemize}
    \item{The word ``must'' indicates an absolute requirement of the
    specification;}

    \item{The word ``should'' indicates behavior that is normally included in
    implementations of the specification, but there may exist valid reasons
    for excluding it in particular circumstances;}

    \item{The word ``may'' indicates purely optional behavior which is
    permitted according to this specification.}
\end{itemize}

\section{Common characteristics}

\subsection{Design goals}

The Transport Protocol, hereafter VTP, is provides a simple means of
transporting VOEvents from authors through brokers to subscribers.

VTP transmits a up to single VOEvent message in each transaction. If multiple
VOEvent messages are to be transmitted, multiple transactions must take place.

VTP is non-transformational on VOEvents being transmitted: the message
delivered to a subscriber should be bit-for-bit identical to that provided by
an author. If an intermediary wishes to modify or annotate the event, they
should not edit the event in transport, but rather generate a new event to
supplement or replace it.

VTP does not provide a transport-level means of annotating or otherwise
embellishing VOEvents, or of providing stream-level metadata.

VTP values simplicity of design and operation to lower the barrier to entry. It
is not intended to meet every use case. As per Section \ref{sec:intro}, it is
anticipated that some VOEvent-based services will require more complex
protocols.

\subsection{Network layer}

VTP operates over TCP (Information Sciences Institute, 1981) connections, and
relies on TCP's guaranteed error-free in-order delivery of data: no checksum
or digest data is included. All messages are sent over the TCP connection
preceded by a 4-byte network-ordered\footnote{As defined by
\citet{Reynolds:1994}; also called ``big-endian'' ordering.} count, followed
immediately by the payload data. The 4-byte count is interpreted as a 32-bit
integer equal to the number of payload bytes following the count bytes. The
payload is considered an opaque collection of bytes at this level\footnote{As
a result, the format of the document being transmitted is opaque to the
transport layer. Therefore both ASCII and UTF-8 are equally supported}.

\subsection{Message format}

The payload of the a VTP message is an XML document. It must consist of an XML
declaration followed by, in order, optional XML comments, a single VOEvent or
Transport element, and more optional XML comments. It must validate against
either the VOEvent XML
schema\footnote{\url{http://www.ivoa.net/xml/VOEvent/VOEvent-v2.0.xsd}} or the
Transport XML schema (Appendix \ref{sec:transportschema}). Transport elements
are used by the VTP system itself and are invisible to end-users: see Section
\ref{sec:transport} for details.

\subsection{Broker behaviour}

Although the simplest broker implementation may simply forward all unique
events it receives, either directly from authors or from other brokers, to all
of its subscribers, this behavior is not required. Instead, the broker may
provide ``value-added'' services which limit how messages are redistributed.
For example, a broker may make arrangements with some or all of its
subscribers to filter the events it receives, and only forward those events
fulfilling some predefined criteria to its subscribers. Similarly, brokers may
limit access to some clients based on various criteria (\S\ref{sec:limit}).

VTP does not provide in-band notification of these per-broker details. For
example, the protocol does not make an author submitting to a filtering broker
aware that their event might not be sent to all of the broker's subscribers,
and, similarly, it does not make a subscriber of a filtering broker aware that
they might not receive a complete set of events. It is the responsibility of
authors and subscribers to ensure that the brokers they use provide the
services they require. Brokers should clearly advertise any added-value
behavior they provide, for example on a website or through the IVOA registry.

\section{Node roles}

The VOEvent network consists of three types of nodes (refer to Fig.
\ref{fig:network}):

\begin{itemize}
    \item{Author}
    \item{Broker}
    \item{Subscriber}
\end{itemize}

The flow of messages is over three types of connections:

\begin{itemize}
    \item{Author to Broker}
    \item{Broker to Subscriber}
    \item{Broker to Broker}
\end{itemize}

Each type of connection is discussed qualitatively below.

\subsection{Author to Broker}

When an author wants to submit a VOEvent message to the network, it opens a
TCP connection to a broker, sends the message, waits for a response from the
broker, and then closes the TCP connection. The response from the broker is a
Transport message.

\subsection{Broker to Subscriber}

When a subscriber wants to receive VOEvent message traffic, it opens a TCP
connection to a broker. This connection is kept open continuously. When the
broker receives a message, it relays a copy of the VOEvent message to all of
the connected subscribers. Thus, a subscriber must continuously listen on the
TCP connection and be prepared to receive VOEvent messages at any time, even
when it is busy processing a previously received VOEvent message. When a
subscriber receives a VOEvent message from its broker, it must respond with an
appropriate Transport message.

\subsection{Broker to Broker}

Traffic between brokers uses the preceding methods. Each broker takes the role
subscriber as far as every other broker is concerned. A broker that wishes to
receive a feed from another broker should connect to that broker's subscriber
port. No special protocol features are needed.

\section{Connection Maintenance}

All connections over which a broker sends VOEvent messages are kept open
continuously. Basic TCP does not provide any dead-peer indication\footnote{
TCP does support a ``keep-alive'' service, but it is not universally available
\citep{Braden:1989}.}. Further, network infrastructure devices might sever a
TCP connection after some period of inactivity. This gives rise to the need
for keep-alive messages. After no more than 90 seconds of inactivity on any
given connection, the broker must send a Transport \texttt{iamalive} message,
to which the subscriber must reply with a copy of that message plus some
optional identification information. For details of the message format, see
Section \ref{sec:transport}.  At both ends of the continuous connection, the
node either expects to receive an \texttt{iamalive} message or expects to
receive the response to its \texttt{iamalive} message. If not seen, the node
should assume that the connection has been lost or the peer is dead. At this
point, the node that was responsible for opening the connection may attempt to
re-initiate it. A geometric back-off algorithm may be used to avoid network
load.

\section{Transport messages}
\label{sec:transport}

Transport messages are XML documents. There are four classes of Transport
message:

\begin{itemize}
\item{\texttt{iamalive} (Connection maintenance);}
\item{\texttt{authenticate} (Authentication request/response);}
\item{\texttt{ack} (VOEvent successful receipt acknowledgement);}
\item{\texttt{nak} (VOEvent unsuccessful receipt acknowledgement).}
\end{itemize}

All Transport messages have the same general syntax, and are defined by the
Transport schema (Appendix \ref{sec:transportschema}). The \texttt{role}
attribute of the root \texttt{<Transport>} element distinguishes between the
types listed above. The connection maintenance and receipt acknowledgement
message types are described in detail in this section; the authentication
message type has a special role which is described in Section
\ref{sec:limit:crypto:subscriber}.

\subsection{\texttt{iamalive} message}

The \texttt{iamalive} message is indicated by a role equal to
\texttt{iamalive}. The \texttt{<Origin>} element contains the IVORN of the
broker which is managing the connection.  The \texttt{<TimeStamp>} element
must contain the UTC date/time at which the message was generated.

\begin{listing*}
\begin{minted}[fontsize=\small,frame=lines]{xml}
<?xml version="1.0" encoding="UTF-8"?>

<trn:Transport role="iamalive" version="1.0"
 xmlns:trn="http://telescope-networks.org/schema/Transport/v1.1"
 xmlns:xsi="http://www.w3.org/2001/XMLSchema-instance"
 xsi:schemaLocation="http://telescope-networks.org/schema/Transport/v1.1
                     http://telescope-networks.org/schema/Transport-v1.1.xsd">
    <Origin>ivo://uk.org.estar/estar.ex#</Origin>
    <TimeStamp>2009-04-09T22:39:06</TimeStamp>
</trn:Transport>
\end{minted}
\caption{Sample \texttt{iamalive} message.}
\label{lst:testmessage}
\end{listing*}


\section{Limiting access}
\label{sec:limit}

\subsection{Cryptographic signatures}

\subsubsection{Subscriber authentication}
\label{sec:limit:crypto:subscriber}

\appendix

\section{Transport schema}
\label{sec:transportschema}

\bibliographystyle{elsarticle-harv}
\bibliography{vtp}

\end{document}
